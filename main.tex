% !TEX TS-program = pdflatex
% !TEX encoding = UTF-8 Unicode

% This is a simple template for a LaTeX document using the "article" class.
% See "book", "report", "letter" for other types of document.

\documentclass[11pt]{article} % use larger type; default would be 10pt

\usepackage[utf8]{inputenc} % set input encoding (not needed with XeLaTeX)
\usepackage[T2A]{fontenc}

%%% Examples of Article customizations
% These packages are optional, depending whether you want the features they provide.
% See the LaTeX Companion or other references for full information.

%%% PAGE DIMENSIONS
\usepackage{geometry} % to change the page dimensions
\geometry{a4paper} % or letterpaper (US) or a5paper or....
% \geometry{margin=2in} % for example, change the margins to 2 inches all round
% \geometry{landscape} % set up the page for landscape
%   read geometry.pdf for detailed page layout information

\usepackage{graphicx} % support the \includegraphics command and options
\usepackage[most]{tcolorbox} % для управления цветом

% Настройка для рамки вокруг текста
\definecolor{block-gray}{gray}{0.90} % уровень прозрачности (1 - максимум)
\newtcolorbox{myquote}{colframe=block-gray,grow to right by=-10mm,grow to left by=-10mm,
boxrule=2pt,boxsep=0pt,breakable} % настройки области с изменённым фоном
% \usepackage[parfill]{parskip} % Activate to begin paragraphs with an empty line rather than an indent

%%% PACKAGES
\usepackage{booktabs} % for much better looking tables
\usepackage{array} % for better arrays (eg matrices) in maths
\usepackage{paralist} % very flexible & customisable lists (eg. enumerate/itemize, etc.)
\usepackage{verbatim} % adds environment for commenting out blocks of text & for better verbatim
\usepackage{subfig} % make it possible to include more than one captioned figure/table in a single float
% These packages are all incorporated in the memoir class to one degree or another...

%%% HEADERS & FOOTERS
\usepackage{fancyhdr} % This should be set AFTER setting up the page geometry
\pagestyle{fancy} % options: empty , plain , fancy
\renewcommand{\headrulewidth}{0pt} % customise the layout...
\lhead{}\chead{}\rhead{}
\lfoot{}\cfoot{\thepage}\rfoot{}

%%% SECTION TITLE APPEARANCE
\usepackage{sectsty}
\allsectionsfont{\sffamily\mdseries\upshape} % (See the fntguide.pdf for font help)
% (This matches ConTeXt defaults)

%%% ToC (table of contents) APPEARANCE
\usepackage[nottoc,notlof,notlot]{tocbibind} % Put the bibliography in the ToC
\usepackage[titles,subfigure]{tocloft} % Alter the style of the Table of Contents
\renewcommand{\cftsecfont}{\rmfamily\mdseries\upshape}
\renewcommand{\cftsecpagefont}{\rmfamily\mdseries\upshape} % No bold!
%%% END Article customizations

\oddsidemargin=0pt

%%% The "real" document content comes below...

\title{Забытый артефакт}
\author{SW Duncan's community}
%\date{} % Activate to display a give date or no date (if empty),
         % otherwise the current date is printed 

\begin{document}
\maketitle %титульная страница
\newpage
\tableofcontents %оглавление
\newpage
\section{В поисках храма}
\begin{myquote}
Планета была похожа на результат столкновения двух капель краски: зеленой и синей. 
В результате на планете образовалось две цветный кляксы неровно притертый друг к другу,
 и не было видно ни одно цвета, кроме синего и зеленого на этой планете от экватора до полюса.
\end{myquote}
\subsection{Воздушный вояж}
\begin{myquote}
Не заходя на посадку, вы решил искать храм по воздуху.
Спустив свой корабь к макушкам исполинских деревев, вы издали заметили несколько золмов из крон.
Летая от одного холма к другому вы находили древние огромные постройки, в текущий момент разрушенные почти до основания.
Однако на последней попытке вам повезло, под деревьями прятлась единственная целая постройка, ваша цель.
\end{myquote}


\begin{myquote}
Не заходя на посадку, вы решил искать храм по воздуху.
Спустив свой корабь к макушкам исполинских деревев, вы издали заметили несколько холмов из крон.
Летая от одного холма к другому вы находили древние огромные постройки, в текущий момент разрушенные почти до основания.
Во время поиска, буквально за одну минуту, вы оказлись в тумане, по сути в не проглядной мгле.
Пошле пиел ветер, нет, ветром это можно было назвать, если спутать льва с домашней кошкой. 
На вас ашел ураган, что так свойственен этой непостоянной планете.
И ураган разрзился громом, отключив у всего корабля электроникиу.
Теперь вы были предоставлены только удаче.

Спустя пару часов в каруселе сильнейшего за вашу жизнь урагана вы разбились рядом со старым зднием.
Корабль больше не подходил для полетов.
Однако, когда небо просветлело, то вы поняли, что находитесь рядом с хрмом, который искали.
Хоть одна хорошая новость.
\end{myquote}
\subsection{С дипломатической миссией к каннибалам}
\begin{myquote}
Медленно приближаясь, поверхность планеты превращалась в монолитные джунгли, создавая ощущение, что человек здесь никогда не был. Каждое дерево, казалось, рослосло более тысячи лет, в обхвате став болье, чем смогли бы вмести в свои руки 10 человек. С могучих толстых веток, больше подходящих дубовым стволам, спускались лианы тощиной в человеческую ногу. А где-то далеко кричало большое животное, спасающееся от хищников. Было понятно лишь одно - это опсное место. 
\end{myquote}

\begin{myquote}
Сквозь сплошной покров начали пробиваться лучи солнца.
По непонятной причине вся територия впереди была безлесной.
Но это никак не значило, что там ничего растительного не было.
Из тонких веток, скрученных стеблей травы, огромных листьев были собраны несколько десятков домиков, средь которых резвилась смуглая ребятня, занимались рукоделием женщины, где-то кучковались мужчины. В центре же деревни стоял камень, похожий на вздернутый вверх палец, который был сверху обильно обагрен застарелой и свежей кровью.
\end{myquote}
\end{document}
